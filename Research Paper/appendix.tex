\section*{Appendix}
\label{Appendix}

Replace all this with your Appendix, if appropriate.

The Appendix contains supplementary material that supports the main text but is not essential for the core understanding of the research. This may include detailed data tables, additional figures, extended mathematical derivations, technical details, or software code that would otherwise clutter the main sections. 


%\vspace{0.5cm}

\begin{center}
	\textbf{Weekly Assignment Overview }
\end{center}

\begin{itemize}
	\item[$\checkmark$] Every week you need to read the required readings (selected articles)
	\item[$\checkmark$] Write two 1-2 page critiques on any 2 of the selected articles for each topic
	\item[$\checkmark$] Use MS word; double spaced, times new roman, font size 12
	\item[$\checkmark$] Note:  A sample critique document is provided on the course website. 
\end{itemize}

%\vspace{2.0cm}

\textbf{Critique Paper Structure:}

%\vspace{0.5cm}
%\par\noindent\rule{\textwidth}{0.4pt}
%\rule{5cm}{0.4pt}
\par\noindent\rule{8.9cm}{0.4pt}

\begin{center}
	\textbf{Title Page}
	\\[5pt]
	A critique of ``$<$title of paper$>$ by $<$author(s)$>$''
	\\[5pt]
	By:  $<$your name, student $\#$, course code, etc.$>$
	\\[1pt]
\end{center}

\hfill \break
\textbf{Summary:}
The section should summarize the purpose, methodology, and main findings of the paper.

\hfill \break
\textbf{Critique:}
This section should critically examine the paper, identify the pros and cons, limitations, and achievements.  Your critique should be objective and factual.  There are many sites that can guide you in writing a critique (e.g.,  
\href{https://www.ucalgary.ca/live-uc-ucalgary-site/sites/default/files/teams/9/critique-or-reviews-of-research-articles-academic-genre.pdf}{University of Calgary}, \href{https://open.umich.edu/sites/default/files/downloads/Topic8Assignment-CritiqueArticle.pdf}{University of Michigan}, etc.)

\hfill \break
\textbf{Questions: }
List 2-4 questions that you have about the paper.  The intention is that you can raise these questions during the Discussion period after the Ubicom Topic Presenters give their talk.


%\par\noindent\rule{8.9cm}{0.4pt}
%\par\noindent\rule{\textwidth}{0.4pt}

\hfill \break
Create two versions for each of your critiques:  
\begin{enumerate}
	\item \textit{Student Authored Version:} Without any writing assistance, create a critique of the academic paper by yourself; and 
	\item \textit{ChatGPT Co-authored version:} Using ChatGPT, ask it to review your critique and provide suggestions for improvement. Reflect on the prompt engineering sessions you had with the instructor and use these suggestions to revise your critque based on your judgement.  Repeat this process up to 3 times.   
\end{enumerate}

\hfill \break
\textbf{Submission Instructions:}

\begin{enumerate}
	\item Submit these two versions to the course website using the following naming convention for your files:
	

	\footnotesize{`Name\_Week\_X\_Critique\_1\_Student\_Authored\_Only.docx'}
	
	`Name\_Week\_X\_Critique\_1\_Student\_ChatGPT\_co\_authored.docx'

	\hfill

	`Name\_Week\_X\_Critique\_2\_Student\_Authored\_Only.docx'
	
	`Name\_Week\_X\_Critique\_2\_Student\_ChatGPT\_co\_authored.docx'

	\normalsize
	\hfill
	
	\item Print out a hardcopy of critiques and bring them to class so that you can refer to it during the discussion portion of the class.  At the end of the class, hand them to the professor.  
\end{enumerate}